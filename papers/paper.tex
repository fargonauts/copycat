\documentclass[a4paper]{article}

%% Language and font encodings
\usepackage[english]{babel}
\usepackage[utf8x]{inputenc}
\usepackage[T1]{fontenc}

%% Sets page size and margins
\usepackage[a4paper,top=3cm,bottom=2cm,left=3cm,right=3cm,marginparwidth=1.75cm]{geometry}

%% Useful packages
\usepackage{amsmath}
\usepackage{graphicx}
\usepackage[colorinlistoftodos]{todonotes}
\usepackage[colorlinks=true, allcolors=blue]{hyperref}

\title{Distributed Parallel Terraced Scan}
%% \title{The Brain as a Distributed System}?
\author{Lucas Saldyt, Alexandre Linhares}

\begin{document}
\maketitle

\begin{abstract}
We investigate FARG architectures in general, and Copycat in particular.  One of the foundations of those models is the \emph{Parallel Terraced Scan}--a psychologically-plausible model that enables a system to fluidly move between different modes of processing.  Previous work has modeled decision-making under Parallel Terraced Scan by using a central variable of \emph{Temperature}. 

\end{abstract}

\section{Introduction: towards `normal science'}

This paper stems from Mitchell's (1993) and Hofstadter \& FARG (1995). The goals of this project are twofold:

 	First, is to create a framework of  `normal science' in FARG architectures. 
    By `normal science' we use the term created by Thomas Kuhn--the collaborative enterprise of furthering understanding within a paradigm. 
    Today, "normal science" is simply not done on FARG architectures (and on most computational cognitive architectures too... see Addyman \& French 2012). 
    Unlike mathematical theories or experiments, which can be replicated by following the materials and methods, computational models generally have dozens of particularly tuned variables, undocumented procedures, multiple assumptions about the users computational environment, etc. 
    It then becomes close to impossible to reproduce a result, or to test some new idea. 
    This paper focuses on the introduction of statistical techniques, reduction of "magic numbers", improvement and documentation of formulas, and proposals for effective human comparison.

    Secondly, we focus on effectively simulating intelligent processes through increasingly distributed decision-making.
    The Parallel Terraced Scan is a major innovation of FARG architectures. 
    It corresponds to the psychologically-plausible behavior of briefly browsing, say, a book, and delving deeper whenever something sparks one's interest. 
    This type of behavior seems to very fluidly change the intensity of an activity based on local, contextual cues. 
    It is found in high-level decisions such as marriage and low-level decisions such as a foraging predator choosing whether to further explore a particular area. 
    Previous FARG models have used a central temperature T to implement this behavior. 
    We explore how to maintain the same behavior while distributing decision-making throughout the system.
    Specifically, we begin by attempting different refactors of the copycat architecture.
    First, we remove temperature destructively, essentially removing any lines of code that mention it, simply to see what effect it has.
    Then, we move toward a surgical removal of temperature, leaving in tact affected structures or replacing them by effective distributed mechanisms.

    We also discuss, in general, the nature of the brain as a distributed system.
    While the removal of a single global variable may initially seem trivial, one must realize that copycat and other cognitive architectures have many central structures.
    This paper explores the justification of these central structures in general.
    Is it possible to model intelligence with them, or are they harmful?
    ...


\bibliographystyle{alpha}
\bibliography{sample}

\end{document}
